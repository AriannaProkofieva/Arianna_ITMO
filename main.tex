\documentclass[12pt]{article}
 \usepackage{amsmath}
 \usepackage{latexsym}
 \usepackage{amsfonts}
 \usepackage[normalem]{ulem}
 \usepackage{soul}
 \usepackage{array}
 \usepackage{amssymb}
 \usepackage{extarrows}
 \usepackage{graphicx}
 \usepackage[backend=biber,
 style=numeric,
 sorting=none,
 isbn=false,
 doi=false,
 url=false,
 ]{biblatex}\addbibresource{bibliography.bib}

  \usepackage{subfig}
 \usepackage{wrapfig}
 \usepackage{txfonts}
 \usepackage{wasysym}
 \usepackage{enumitem}
 \usepackage{adjustbox}
 \usepackage{ragged2e}
 \usepackage[svgnames,table]{xcolor}
 \usepackage{tikz}
 \usepackage{longtable}
 \usepackage{changepage}
 \usepackage{setspace}
 \usepackage{hhline}
 \usepackage{multicol}
 \usepackage{tabto}
 \usepackage{float}
 \usepackage{multirow}
 \usepackage{makecell}
 \usepackage{fancyhdr}
 \usepackage[toc,page]{appendix}
 \usepackage[hidelinks]{hyperref}
 \usetikzlibrary{shapes.symbols,shapes.geometric,shadows,arrows.meta}
 \tikzset{>={Latex[width=1.5mm,length=2mm]}}
 \usepackage{flowchart}\usepackage[paperheight=11.69in,paperwidth=8.27in,left=1.18in,right=0.39in,top=0.79in,bottom=0.79in,headheight=1in]{geometry}
 \usepackage[utf8]{inputenc}
 \usepackage[russian]{babel}
 \usepackage[T1,T2A]{fontenc}
 \TabPositions{0.5in,1.0in,1.5in,2.0in,2.5in,3.0in,3.5in,4.0in,4.5in,5.0in,5.5in,6.0in,6.5in,}

  \urlstyle{same}

 
   %%%%%%%%%%%%  Set Depths for Sections  %%%%%%%%%%%%%%

  % 1) Section
 % 1.1) SubSection
 % 1.1.1) SubSubSection
 % 1.1.1.1) Paragraph
 % 1.1.1.1.1) Subparagraph

 
  \setcounter{tocdepth}{5}
 \setcounter{secnumdepth}{5}

 
   %%%%%%%%%%%%  Set Depths for Nested Lists created by \begin{enumerate}  %%%%%%%%%%%%%%

 
  \setlistdepth{9}
 \renewlist{enumerate}{enumerate}{9}
 		\setlist[enumerate,1]{label=\arabic*)}
 		\setlist[enumerate,2]{label=\alph*)}
 		\setlist[enumerate,3]{label=(\roman*)}
 		\setlist[enumerate,4]{label=(\arabic*)}
 		\setlist[enumerate,5]{label=(\Alph*)}
 		\setlist[enumerate,6]{label=(\Roman*)}
 		\setlist[enumerate,7]{label=\arabic*}
 		\setlist[enumerate,8]{label=\alph*}
 		\setlist[enumerate,9]{label=\roman*}

  \renewlist{itemize}{itemize}{9}
 		\setlist[itemize]{label=$\cdot$}
 		\setlist[itemize,1]{label=\textbullet}
 		\setlist[itemize,2]{label=$\circ$}
 		\setlist[itemize,3]{label=$\ast$}
 		\setlist[itemize,4]{label=$\dagger$}
 		\setlist[itemize,5]{label=$\triangleright$}
 		\setlist[itemize,6]{label=$\bigstar$}
 		\setlist[itemize,7]{label=$\blacklozenge$}
 		\setlist[itemize,8]{label=$\prime$}

  \pagenumbering{gobble}
 \setlength{\topsep}{0pt}\setlength{\parskip}{9.96pt}
 \setlength{\parindent}{0pt}

   %%%%%%%%%%%%  This sets linespacing (verticle gap between Lines) Default=1 %%%%%%%%%%%%%%

 
  \renewcommand{\arraystretch}{1.3}

 
  %%%%%%%%%%%%%%%%%%%% Document code starts here %%%%%%%%%%%%%%%%%%%%

 
 
  \begin{document}
 \begin{Center}
 ИТМО
 \end{Center}\par

  \begin{Center}
 Институт математики
 \end{Center}\par
 \vspace{\baselineskip}
 ОТЧЕТ \\
 ЗАЩИЩЕН С ОЦЕНКОЙ\par

  ПРЕПОДАВАТЕЛЬ\par

 
 
  %%%%%%%%%%%%%%%%%%%% Table No: 1 starts here %%%%%%%%%%%%%%%%%%%%

 
  \begin{table}[H]
  			\centering
 \begin{tabular}{p{2.05in}p{0.0in}p{1.76in}p{-0.01in}p{1.89in}}
 %row no:1
 \multicolumn{1}{p{2.05in}}{\Centering {Преподаватель}} & 
 \multicolumn{1}{p{0.0in}}{} & 
 \multicolumn{1}{p{1.76in}}{} & 
 \multicolumn{1}{p{-0.01in}}{} & 
 \multicolumn{1}{p{1.89in}}{\Centering {М.Д.Поляк}} \\
 \hhline{-~-~-}
 %row no:2
 \multicolumn{1}{p{2.05in}} {\Centering{\fontsize{10pt}{12.0pt}\selectfont  {должность, уч. степень, звание}}} & 
 \multicolumn{1}{p{0.0in}}{} & 
 \multicolumn{1}{p{1.76in}} {\Centering{\fontsize{10pt}{12.0pt}\selectfont  {подпись, дата}}} & 
 \multicolumn{1}{p{-0.01in}}{} & 
 \multicolumn{1}{p{1.89in}} {\Centering{\fontsize{10pt}{12.0pt}\selectfont  {инициалы, фамилия}}} \\
 \hhline{~~~~~}

  \end{tabular}
  \end{table}

 
  %%%%%%%%%%%%%%%%%%%% Table No: 1 ends here %%%%%%%%%%%%%%%%%%%%

 
  \vspace{\baselineskip}

 
  %%%%%%%%%%%%%%%%%%%% Table No: 2 starts here %%%%%%%%%%%%%%%%%%%%

 
  \begin{table}[H]
  			\centering
 \begin{tabular}{p{6.49in}}
 %row no:1
 \multicolumn{1}{p{6.49in}}{\fontsize{14pt}{16.8pt}\selectfont {\section*{\Centering {ОТЧЕТ О ЛАБОРАТОРНОЙ РАБОТЕ №0}}}}\\
 \hhline{~}
 %row no:2
 \multicolumn{1}{p{6.49in}}{\section*{\Centering {Знакомство с системой контроля версий git и сервисом GitHub}}
 } \\
 \hhline{~}
 %row no:3
 \multicolumn{1}{p{6.49in}}{\subsubsection*{\Centering {по курсу: ОСНОВЫ СТАТИСТИКИ}}
 } \\
 \hhline{~}
 %row no:4
 \multicolumn{1}{p{6.49in}}{} \\
 \hhline{~}
 %row no:5
 \multicolumn{1}{p{6.49in}}{} \\
 \hhline{~}

  \end{tabular}
  \end{table}

 
  %%%%%%%%%%%%%%%%%%%% Table No: 2 ends here %%%%%%%%%%%%%%%%%%%%

  РАБОТУ ВЫПОЛНИЛ\par

 
 
  %%%%%%%%%%%%%%%%%%%% Table No: 3 starts here %%%%%%%%%%%%%%%%%%%%

 
  \begin{table}[H]
  			\centering
 \begin{tabular}{p{1.3in}p{1.0in}p{-0.04in}p{1.63in}p{-0.04in}p{1.63in}}
 %row no:1
 \multicolumn{1}{p{1.3in}}{СТУДЕНТ ГР. №} & 
 \multicolumn{1}{p{1.0in}}{\Centering {R4160}} & 
 \multicolumn{1}{p{-0.04in}}{} & 
 \multicolumn{1}{p{1.63in}}{} & 
 \multicolumn{1}{p{-0.04in}}{} & 
 \multicolumn{1}{p{1.63in}}{\Centering {A.Ф.Прокофьева}} \\
 \hhline{~-~-~-}
 %row no:2
 \multicolumn{1}{p{1.3in}}{} & 
 \multicolumn{1}{p{1.0in}}{} & 
 \multicolumn{1}{p{-0.04in}}{} & 
 \multicolumn{1}{p{1.63in}} {\Centering{\fontsize{10pt}{12.0pt}\selectfont  {подпись, дата}}} & 
 \multicolumn{1}{p{-0.04in}}{} & 
 \multicolumn{1}{p{1.63in}} {\Centering{\fontsize{10pt}{12.0pt}\selectfont  {инициалы, фамилия}}} \\
 \hhline{~~~~~~}

  \end{tabular}
  \end{table}

 
  %%%%%%%%%%%%%%%%%%%% Table No: 3 ends here %%%%%%%%%%%%%%%%%%%%

 
  \vspace{\baselineskip}
 \vspace{\baselineskip}
 \vspace{\baselineskip}
 \vspace{\baselineskip}
 \vspace{\baselineskip}
 \begin{Center}
 Санкт-Петербург \the\year{}
 \end{Center}\par

 %==================== НАЧАЛО СОДЕРЖИМОГО ОТЧЕТА ====================

 \newpage
 \pagenumbering{arabic}

Ссылка на репозиторий: \url{https://github.com/AriannaProkofieva/Arianna_ITMO/tree/main}

 \section*{Задание 1.}

 Цель: поразмышлять о своих целях и задачах на текущий семестр
 
 \begin{itemize}
    \item С помощью веб-интерфейса GitHub создать файл goals.md в своем репозитории.
    \item Используя язык разметки markdown, создать заголовки "Мой опыт" и "Мои цели".
    \item Написать один абзац (1-5 предложений) о своем опыте по разработке программного обеспечения. Добавить работающую ссылку в написанный текст (например, на профиль в GitHub).
    \item Создать нумерованный список из 2 или 3 целей, которых планируете достичь в рамках данного курса.
    \item Создать ненумерованный список из 3-9 задач, над которыми собираетесь работать в рамках данного курса для достижения поставленных целей.
    \item Создать папку resources в корне репозитория. Загрузить в эту папку изображение (.png/.jpg/.jpeg) и добавить его в .md файл.
    \item Не забыть закоммитить файлы в свой репозиторий на GitHub.
    \item Проверить, что созданный .md файл отформатирован и отображается при предпросмотре именно так, как планировалось.
 \end{itemize}

 \subsection*{Ход работы:}

 Созданы файлы:
 \begin{itemize}
     \item \textbf{goals.md} -- файл с личными целями и опытом
     \item Папка \textbf{resources/} с изображением
 \end{itemize}
 
\begin{figure}[H]
    \centering
    \includegraphics[width=0.9\textwidth]{image1.png}
    \caption{Рисунок 1 - Содержимое файла goals.md}
\end{figure}


 %==================== ЗАДАНИЕ 2 ====================

 \section*{Задание 2.}
 Цель: научиться использовать git из командной строки
 \begin{itemize}
    \item Ознакомиться с инструкцией по работе с git и github из командной строки.
    \item Клонировать свой репозиторий на локальный компьютер используя команду git clone <ссылка на репозиторий> из командной строки.
    \item Создать новый файл info.md в локальной копии репозитория.
    \item Записать в этот файл свое ФИО (полностью), название используемой операционной системы, текущие дату и время, - каждый элемент с новой строки (всего получится 3 строки).
    \item Выполнить команды git add info.md и git commit в командной строке.
    \item Ввести краткое описание коммита.
    \item Выполнить команду git push.
    \item Создать папку data в корне репозитория и в ней файл dataset.csv. Создать файл .gitignore, в который добавить строку data/. Далее закоммитить .gitignore и попробовать закоммитить data/dataset.csv, после чего отправить коммиты в репозиторий на github.com с помщью git push. Объяснить полученный результат в отчёте.
 \end{itemize}

 \subsection*{Ход работы:}

 \subsubsection*{1. Клонирование репозитория}

 Команда:
 \begin{verbatim}
 git clone https://github.com/AriannaProkofieva/AriannaProkofieva_ITMO.git
 \end{verbatim}
 
 Результат: репозиторий успешно скопирован на локальный компьютер.

 \subsubsection*{2. Создание файла info.md, коммит и push}

 Команды:
 \begin{verbatim}
 notepad info.md
 git add info.md
 git commit
 git push
 \end{verbatim}

 Результат: файл успешно добавлен в репозиторий на GitHub.

 \begin{figure}[H]
    \centering
    \includegraphics[width=0.9\textwidth]{image2.png}
    \caption{Рисунок 2 - Содержимое файла info.md}
\end{figure}

 \subsubsection*{3. Работа с .gitignore}

 Создан файл .gitignore с содержимым:
 \begin{verbatim}
 data/
 \end{verbatim}

 Затем создана папка data с файлом dataset.csv.

 \subsubsection*{Результат попытки закоммитить data/dataset.csv}

 Команда:
 \begin{verbatim}
 git add data/dataset.csv
 \end{verbatim}

 Результат в консоли:
 \begin{verbatim}
 The following paths are ignored by one of your .gitignore files:
 data
 hint: Use -f if you really want to add them.
 hint: Disable this message with "git config set advice.addIgnoredFile false"
 \end{verbatim}

 \subsection*{Объяснение результата}

 Файл \textbf{data/dataset.csv} не был добавлен в коммит, потому что папка \textbf{data/} указана в файле \textbf{.gitignore}. 
 Git игнорирует все файлы и папки, перечисленные в этом файле.

  \begin{figure}[H]
    \centering
    \includegraphics[width=0.9\textwidth]{image3.png}
    \caption{Рисунок 3 - Содержимое файла .gitignore}
\end{figure}

 %==================== ЗАКЛЮЧЕНИЕ ====================

 \section*{Заключение}

 В ходе выполнения лабораторной работы были освоены навыки работы с веб-интерфейсом GitHub, написания документации на языке Markdown, а также изучены основные команды системы контроля версий Git и принципы использования файла .gitignore.

\end{document}
